
            
            % métadonnées
            \author{Paul, Hector KERVEGAN}
            \title{Transformation XSL d'une édition critique en XML-TEI vers LaTeX: les Chroniques de Jean Froissart, SHF 306: "La négociation du mariage du duc de Berry"}
            \date{25.04.2022}
            
            
            % chouettes paquets utilisés
            \documentclass[12pt, a4paper]{article}
            \usepackage[utf8x]{inputenc}
            \usepackage[T1]{fontenc}
            \usepackage{fontspec}
            \usepackage{lmodern}
            \usepackage{graphicx}
            \usepackage[french]{babel}
            \usepackage{reledmac}
            \usepackage[switch, modulo]{lineno}
            \usepackage[toc]{appendix}
            
            % hyperref
            \usepackage{hyperref}
            \hypersetup{pdfauthor={Paul, Hector KERVEGAN}, pdftitle={Transformation XSL d'une édition critique en XML-TEI vers LaTeX: les Chroniques de Jean Froissart, SHF 306: "La négociation du mariage du duc de Berry"}, pdfsubject={Transformation XSL vers LaTeX}, pdfkeywords={édition critique}{XSL}{XML-TEI}{LaTeX}{Jean Froissart}}
            
            
            
            % définition de commandes spécifiques pour l'apparat
            % variation de 1er degré
            \newcommandx{\variant}[2][1,usedefault]{\Afootnote[#1]{#2}}
            % groupe de témoins
            \newcommandx{\group}[2][1,usedefault]{\Bfootnote[#1]{#2}}
            % sous-variation (un rdg dans un apparat interne)
            \newcommandx{\subvariant}[2][1,usedefault]{\Cfootnote[#1]{#2}}
            % détails non textuels sur le témoin
            \newcommandx{\explan}[2][1,usedefault]{\Dfootnote[#1]{#2}}
            
            \Xarrangement[A]{paragraph}
            \Xparafootsep{$\parallel$~}
            
            \begin{document}
            
            
            \begin{titlepage}
            \begin{center}
            \large
            Paul, Hector KERVEGAN
            
            \Large
            \vfill
            \textbf{Transformation XSL d'une édition critique en XML-TEI vers \LaTeX}
            \\~\\
            \textbf{\textit{CHRONIQUES} DE JEAN FROISSART, SHF 3A-306 : "LA NÉGOCIATION DU MARIAGE DU DUC DE BERRY"}
            \vfill
            
            \large	
            \vfill Devoir réalisé pour le master 2 TNAH de l'École nationale des Chartes - Avril 2022
            \end{center}
            \end{titlepage}
            
            
             \section{Présentation du projet} \subsection{Présentation de l'encodage XML-TEI et de sa source en ligne: l'Online Froissart} \subsubsection{À propos du présent encodage} La présente édition critique d'un chapitre des \textit{Chroniques} de Jean Froissart a été encodée en XML-TEI par Paul, Hector Kervegan. L'encodage s'appuie sur une édition en ligne des \textit{Chroniques} réalisée dans le cadre du projet Online Froissart.\\ \indent L'encodage en XML-TEI a été réalisé dans le cadre l'évaluation du cours de TEI du master 2 TNAH de l'École nationale des Chartes (Paris, France). L'encodage a été finalisé le 31 janvier 2022. Le présent document résulte d'une transformation XSL vers \LaTeX réalisée en avril 2022 dans le cadre de l'évaluation du cours d'XSLT de ce master. \subsubsection{À propos du projet Online Froissart} Sélection, alignement des témoins et première édition en ligne par \href{https://www.dhi.ac.uk/onlinefroissart}{\textit{The Online Froissart. A digital edition of the Chronicles of Jean Froissart}}. La source de l'encodage est disponible à \href{https://www.dhi.ac.uk/onlinefroissart/browsey.jsp?f=b&pb2=Ber-3_SHF_3A-307&disp2=shf&GlobalWord=0&div2=ms.f.transc.Ber-3&div1=ms.f.transc.Bre-3&div0=ms.f.transc.Ant-3&panes=3&GlobalMode=shf&img0=&disp0=shf&disp1=shf&pb1=Bre-3_SHF_3A-307&img2=&GlobalShf=3A-306&pb0=Ant-3_SHF_3A-307&img1=}{cette adresse}.\\~\\ \noindent Responsables du projet Online Froissart : \begin{itemize}\item{ Peter Ainsworth } \item{ Godfried Croenen } \end{itemize} \noindent Institutions partenaires du Online Froissart : \begin{itemize}\item{University of Sheffield} \item{University of Liverpool} \item{AHRC. Art and Humanities Research Council} \item{The Digital Humanities Institute} \item{ATLIF. Laboratoire Analyse et Traitement Informatique de la Langue Française} \item{CNRS. Centre National de la Recherche Scientifique} \end{itemize} \subsection{Description des témoins} \noindent \textbf{Manuscrit de Berlin (\texttt{brl})}\\ \indent Le manuscrit de Berlin (Allemagne) a pour identifiant \textbf{\texttt{brl}}. Il est conservé dans le Département des Manuscrits de la Staatsbibliothek zu Berlin - Preussischer Kulturbesitz. Il a pour cote "Dep. Breslau 1" et est également connu sous le nom de \textit{Froissart de Breslau}. Il est écrit en Moyen français en 1468-1469Le témoin de Berlin est le plus ancien des trois témoins utilisés dans cette édition critique. Il est donc retenu comme présentant la \textbf{leçon principale}. Son contenu constitue le corps de texte de l'édition \LaTeX, et le \texttt{lem} de l'édition TEI.\\ \indent Le manuscrit est écrit sur parchemin. Il est d'une longueur de 343 folio. Ces informations ont été proviennent de \href{http://medium-avance.irht.cnrs.fr/ark:/63955/md44pk02gf53}{Medium - Répertoire des manuscrits reproduits ou recensés}.\\~\\\noindent \textbf{Manuscrit d' Anvers (\texttt{atw})}\\ \indent Le manuscrit d'Anvers (Belgique) a pour identifiant \textbf{\texttt{atw}}. Il est conservé dans la Collection des manuscrits du Museum Plantin-Moretus. Il a pour cote "M 15.6". Il est écrit en Moyen français en 1475-1485.\\ \indent Le manuscrit est écrit sur parchemin. Il est d'une longueur de 350 folio. Ces informations ont été proviennent de la \href{https://anet.be/record/opacmpm/c:lvd:14057240/F}{Anet Departement Bibliotheek}.\\~\\\noindent \textbf{Manuscrit de Berne (\texttt{brn})}\\ \indent Le manuscrit de Berne (Suisse) a pour identifiant \textbf{\texttt{brn}}. Il est conservé dans les Archives de la Burgerbibliothek Bern - Bibliothèque de la Bourgeoisie. Il a pour cote "Cod. A 14". Il est écrit en Moyen français en 1470-1490 Suisse Berne Burgerbibliothek Bern - Bibliothèque de la Bourgeoisie Archives Section Bongarsiana/Codices Cod. A 14 Moyen français parchemin II + 336 folio 44 34,5 10 1470-1490 Anciennement propriété du Comte Alexandre de Donha, bourgeois de Berne, il en fait don à la bibliothèque en 1697 Bibliothèque de la Bourgeoisie - Catalogue en ligne des Archives .\\ \indent Le manuscrit est écrit sur parchemin. Il est d'une longueur de II + 336 folio .Il mesure 44 centimètres de hauteur, 34,5 cm de largeur et 10 cm de profondeur. Ces informations ont été proviennent de la \href{http://katalog.burgerbib.ch/detail.aspx?ID=147296}{Bibliothèque de la Bourgeoisie - Catalogue en ligne des Archives}. \subsection{Principes d'encodage suivis pour l'édition en XML-TEI} L'alignement des témoins a été réalisé avec l'aide de l'algorithme de Dekker de \href{https://collatex.net/}{Collatex} . Cet alignement a été repris ensuite pour constituer l'édition critique qui suit.L'édition critique et l'alignement des témoins cherchent à permettre l'étude la plus précise possible de la variance et des similitudes entre les versions: \begin{itemize} \item{La structuration du texte en paragraphes (\texttt{<p>}) a été abandonnée : les trois textes sont insérés dans un conteneur \texttt{<ab>} et les changements de paragraphes sont signalés par des \texttt{<milestone>}, pour permettre un alignement des témoins phrase par phrase} \item{Pour cet encodage, on considère que les différences au niveau de la structure de la phrase sont plus importantes que les différences orthographiques. L'encodage suit ce principe: quand la même phrase est dans un ordre différent dans différents témoins, cette phrase est encodée comme un élément entier, plutôt que de rassembler les mêmes mots de la phrase dans un seul élément \texttt{<app>}(ce qui permettrait de mettre en avant les différences orthographiques entre ces mêmes mots).} \item{Pour permettre l'étude de variance la plus précise possible, l'encodage suit un principe de modularité: lorsque 2 témoins sont semblables et diffèrent du 3e témoin, alors ces 2 témoins sont encodés dans un seul élément \texttt{<rdg>}. Les variations au sein de ces 2 témoins sont signalées dans un \texttt{<app>} à l'intérieur du \texttt{<rdg>}. L'édition est donc construite en variations/sous-variations.} \item{L'élément \texttt{<rdgGrp>} est utilisé pour rassembler deux lectures très proches ; cependant, pour privilégier l'étude de la variation entre les témoins et pour coller au plus près de ces variations, l'utilisation de \texttt{<rdg>} rassemblant plusieurs témoins et contenant un ou plusieurs \texttt{<app>} (comme décrit au dessus) est privilégiée.} \end{itemize} Pour chaque élément \texttt{<app>}, un attribut "\texttt{@corresp"} est utilisé pour préciser la nature de la différence entre les témoins. Les valeurs possible pour cet attribut sont: \begin{itemize} \item{\texttt{orth} : différence orthographique} \item{\texttt{punct} : différence de ponctuation} \item{\texttt{sentence-order} : différence dans l'ordre des mots dans un bloc de texte} \item{\texttt{word-diff} : des mots différents sont utilisés dans un même bloc de texte dans les différents témoins} \item{\texttt{add-words} : un témoin présente un ou plusieurs mots qui ne se trouvent pas dans les autres} \item{\texttt{text-segmentation} : la séparation du texte en paragraphes et pages est différente d'un témoin à l'autre} \item{\texttt{ornementation} : la décoration du manuscrit varie d'un témoin à l'autre} \end{itemize} Pour mieux aligner les textes et mettre en avant les variations entre les témoins, les paragraphes ne sont pas contenus dans des \texttt{<p>}. Le passage d'un paragraphe à l'autre est signalé par un \texttt{<milestone unit="tei:p"/>} et par un \texttt{<lb/>}. Cette méthode est suggérée par Beshero-Bondar, Cayless, Vigilanti (2019: 3.10) \footnote{ Beshero-Bondar, Cayless, Viglianti , "Document Modeling with the TEI Critical Apparatus". \href{http://bit.ly/crit-app-panel}{Disponible en ligne} (consulté le 18.04.22).} \subsection{Présentation du texte encodé} \subsubsection{À propos des \textit{Chroniques} de Froissart} Datant du XIVe siècle, les \textit{Chroniques} de Froissart sont écrites en moyen français. Elles couvrent les années 1322 à 1400 et décrivent les évènements de la première moitié de la Guerre de Cent Ans. Ce récit en prose s'ouvre sur les événements qui ont précédé la déposition d'Édouard II en 1326 et couvre la période allant jusqu'à 1400.\\ \indent \subsubsection{À propos de l'extrait encodé} Ce passage est extrait du troisième volumes des Chroniques de Froissart. Dans cet extrait, Jean Froissart rapporte la manière dont le duc Jean de Berry a organisé son second mariage avec Jeanne, comtesse d’Auvergne et de Boulogne. Dès la mort de sa première femme, le duc a organisé des négociations avec le comte Gaston de Foix-Béarn, surnommé Fébus. Ce dernier refuse de laisser le duc lui prendre sa protégée sans négociation, à cause de de querelles familiales. Afin de convaincre Fébus, le duc de Berry cherche à trouver un accord ; pour ce faire, il mobilise tout son important cercle d'influence, dont le duc de Bourgogne et même Charles VI le roi de France.Les messagers du duc et ses proches se lançent donc dans un voyage pour rencontrer le conte de Foix, s'arrêtant à Avignon chez le pape Clément. Connu pour sa finesse politique, Fébus s'est servi de Jeanne afin d'obtenir les bonnes grâces du duc. Il accepte donc de céder la jeune femme au duc, pour obtenir ses bonnes grâces.Ce court réçit est, comme l'ensemble des Chroniques, est un témoignagne du climat politique de la Guerre de Cent Ans, mais aussi des mœurs du XVe siècle : le texte décrit la façon dont les mariages s'organisent durant le Moyen-Âge tardif ; les réactions du roi de France et du duc de Bourgogne quant à différence d'âge du duc et de sa future épouse donnent à voir l'organisation des relations entre les genres au XVe siècle. \subsubsection{Index des personnages} \subsubsection{Index des lieux} \pagebreak 
            
            
            \section{Édition critique}
            \beginnumbering
            \linenumbers
            \pstart    
            
             \edtext{ }{\explan{La leçon principale débute au folio 370 v.}} Le duc de \edtext{Berry}{\variant{Berri atw}}\edtext{,}{\variant{$\phi$ brn}} madame \edtext{Jehanne}{\variant{Jehenne brn}} \edtext{d'Armeignach}{\variant{d'Ermignac atw; d'Armignac brn}} \edtext{sa premiere femme trespassee}{\group{trespassee, sa premiere femme atw; trespassee sa premiere brn}}\edtext{,}{\variant{$\phi$ brn}} avoit grant \edtext{ymagination}{\variant{ymaginacion atw brn}}\edtext{. Et bien le}{\variant{et bien le atw brn}} \edtext{moustra}{\variant{monstra atw brn}} que secondement il \edtext{fust}{\variant{feust atw; fut brn}} \edtext{remarié}{\variant{marié atw brn}}. Car \edtext{tantost que}{\group{tost comme atw; trestost comme brn}} il \edtext{pot}{\variant{peut atw brn}} veoir \edtext{comment il}{\variant{qu'il atw brn}} avoit \edtext{failly}{\variant{faily atw; failli brn}} \edtext{a la fille de Castille,}{\variant{$\phi$ atw brn}} il ne fut oncques \edtext{aise}{\variant{avisé atw}} ne \edtext{n'ot repos}{\group{eut atw; eust brn}}\edtext{. Mais}{\variant{mais brn}} \edtext{}{\variant{il atw brn}} mist \edtext{en oeuvre clers}{\group{clers en oeuvre atw; clers en euvre brn}} et \edtext{messagiers}{\variant{messaigiers atw brn}} pour \edtext{envoier}{\variant{envoyer atw brn}} devers le conte \edtext{Gascoing}{\variant{Gascon atw brn}} de \edtext{Fois}{\variant{Foix atw brn}}\edtext{,}{\variant{$\phi$ brn}} \edtext{quy}{\variant{qui atw brn}} avoit en garde \edtext{la fille}{\variant{le filz brn}} au conte Jehan de \edtext{Boulongne}{\variant{Bouloingne atw}} et avoit \edtext{ja}{\variant{$\phi$ atw brn}} eue plus de \edtext{noeuf}{\variant{neuf atw; IX brn}} ans. Et pour tant que le duc de \edtext{Berry}{\variant{Berri atw}} ne pouoit \edtext{a ce mariage parvenir}{\variant{parvenir a ce mariage atw}} fors par \edtext{les dangiers}{\variant{le dangier atw brn}} du conte de \edtext{Fois}{\variant{Foix atw brn}}. Car au fort le dit conte ne pour pere \edtext{}{\variant{, atw brn}} ne pour mere \edtext{}{\variant{, atw brn}} \edtext{ne}{\variant{$\phi$ brn}} pour pape \edtext{}{\variant{, atw brn}} ne pour amy que la damoiselle eust\edtext{,}{\variant{$\phi$ brn}} il n'en \edtext{euist}{\variant{eust atw brn}} riens fait \edtext{\edtext{, se bienne lui feust}{\group{s'il ne lui fut bien atw; s'il ne lui fust bien brn}}}{\explan{Changement de page dans la leçon principale : passage au folio 371 r.}} venu a \edtext{sa}{\variant{$\phi$ atw brn}} plaisance. Il en parla au roy de France, son nepveu, et au duc de \edtext{Bourgoingne}{\variant{Bourgongne brn}}, son frere\edtext{. Et leur}{\variant{, et leur atw brn}} \edtext{requist moult}{\variant{pria atw brn}} affectueusement \edtext{que ilz}{\variant{qu'ilz atw brn}} s'en voulsissent chargier\edtext{, avecques luy et embesoingnier}{\group{et embesoignier avecques lui atw; et embesongnier avecques lui brn}}. Le roy de France en \edtext{ot}{\variant{eut atw brn}} bon \edtext{rys,}{\variant{ris atw brn}} pour tant que le duc de Berry estoit ja tout \edtext{anchien,}{\variant{ancien atw brn}} et \edtext{luy}{\variant{lui atw brn}} \edtext{dist}{\variant{dit atw}}:\edtext{"Beaulx oncles}{\variant{"Bel oncle atw brn}}, que feréz vous \edtext{de une si jenne femme }{\group{d'unefillette atw; d'une tellefillette brn}} \edtext{? Elle }{\variant{qui atw}} n'a que \edtext{douze}{\variant{XII atw brn}} ans et vous en avéz \edtext{soixante}{\variant{LX atw brn}}\edtext{.}{\variant{$\phi$ atw}} Par ma foy, \edtext{c'est}{\variant{il me semble atw}} grant \edtext{folie pour}{\variant{follie de atw}} vous de penser a telle \edtext{besoigne}{\variant{chose atw brn}}. \edtext{Faittes}{\variant{Faictes brn}} \edtext{ent}{\variant{en atw}} parler pour Jehan \edtext{}{\variant{, atw brn}} beau cousin vostre filz, \edtext{quy}{\variant{qui atw brn}} est \edtext{jenne}{\variant{jeune atw}} et a venir\edtext{. La}{\variant{, la atw}} chose est \edtext{trop}{\variant{$\phi$ atw brn}} mieulx pareille a \edtext{\edtext{luy}{\variant{lui atw brn}}}{\explan{Enluminure ou miniature, signalée mais non consultable depuis Online Froissart}} que a vous." "Monseigneur,"\edtext{respondi}{\variant{respondit atw brn}} le duc de \edtext{Berry}{\variant{Berri atw}}, "on en a parlé\edtext{,}{\variant{$\phi$ atw brn}} mais le conte de \edtext{Foiz}{\variant{Foix atw brn}} n'y \edtext{a voulu}{\group{vault atw; veult brn}} entendre pour la cause de ce que mon filz descend de ceulx \edtext{d'Armeignach. Et}{\variant{d'Armignac et atw; de Armignac brn}} ilz sont en guerre et en hayne et ont esté \edtext{long temps}{\variant{longtemps brn}} ensemble. Se la fille est \edtext{jenne}{\variant{jeune atw}}, je \edtext{la espargneray}{\group{l'espareigneray atw; l'espargneray brn}} \edtext{trois}{\variant{troiz brn}} ou quatre ans\edtext{, voire si longuement}{\variant{tant atw brn}} qu'elle sera femme \edtext{parfaitte et}{\variant{parfaicte brn}} \edtext{fourmee}{\variant{formee atw; $\phi$ brn}}." Voire,"\edtext{respondy}{\variant{dit atw; respondit brn}} le roy,\edtext{" beaulx oncles }{\variant{$\phi$ atw; "bel oncle brn}} \edtext{mais}{\variant{maiz atw}} elle ne vous \edtext{espargnera}{\variant{espareignera atw}} pas." Et puis dist le roy tout en \edtext{ryant}{\variant{riant atw brn}}:\edtext{"Or ça, beaulx oncles }{\variant{"Bel oncle atw brn}}, \edtext{puisque}{\variant{puis que atw brn}} nous veons que \edtext{si tres}{\variant{tant atw; si brn}} grande affection vous y avéz \edtext{}{\variant{, atw}} nous y \edtext{entendrons}{\variant{entenderons brn}} \edtext{moult}{\variant{$\phi$ atw brn}} voulentiers." \pend\pstart \edtext{Depuis}{\explan{Changement de paragraphe dans le témoin principal.}} ne demoura \edtext{gaires}{\variant{guerres atw}} de temps que le roy ordonna le sire de la Riviere, messire \edtext{Burel}{\variant{Brueil atw brn}}, son souverain \edtext{chevallier}{\variant{$\phi$ atw brn}} \edtext{et}{\variant{$\phi$ brn}} maistre d'ostel et \edtext{chambellan}{\variant{chambrelan atw; chambrelain brn}}\edtext{,}{\variant{$\phi$ atw}} pour \edtext{aler}{\variant{aller atw}} en ce voyage\edtext{. Et avecques}{\variant{et avecques atw brn}} \edtext{luy}{\variant{lui atw brn}} le conte \edtext{d'Assy}{\variant{d'Acy atw brn}}. Et le duc de \edtext{Bourgoingne}{\variant{Bourgongne brn}} y ordonna pour \edtext{luy}{\variant{lui atw brn}} et en son nom l'evesque \edtext{d'Authun}{\variant{d'Autun brn}} et messire \edtext{Guillemme}{\variant{Guillaume atw; Guillame brn}} de la Trimoulle \edtext{et}{\variant{. Et atw brn}} le duc de Berry y ordonna \edtext{et}{\variant{$\phi$ atw}} en \edtext{prya}{\variant{pria atw brn}} le conte Jehan de Xancerre \edtext{ung}{\variant{, un brn}} grant\edtext{, sage}{\variant{saige atw brn}} et vaillant \edtext{chevallier}{\variant{chevalier atw}}. \pend\pstart \edtext{ }{\explan{Changement de paragraphe dans le témoin principal.}} Ces \edtext{cinq}{\variant{cincq atw}} seigneurs pour venir \edtext{devers}{\variant{vers atw brn}} le conte de \edtext{Fois}{\variant{Foix atw brn}} et \edtext{}{\variant{pour atw}} requerir \edtext{en}{\variant{ou atw brn}} nom \edtext{de mariage}{\variant{}} pour le duc de Berry ceste \edtext{jenne}{\variant{jeune atw}} dame \edtext{}{\variant{a mariage atw}}\edtext{,}{\variant{$\phi$ brn}} se partirent de leurs lieux et se devoient trouver en Avignon \edtext{deléz le pape comme ils firent. Ce pape Clement, quy cousin estoit du pere a la damoiselle, les retint la bien quinze jours}{\group{ainsi comme ilz deléz le pape Clementbien XV jours, qui cousin germain estoit a ladamoiselle atw; aussi comme ilz delez le pape Clementbien XV jours, qui cousin germain estoit a ladamoiselle, brn}}\edtext{. Et environ la}{\variant{, et environ la atw brn}} \edtext{Chandelleur}{\variant{Chandeller atw; Chandeleur brn}} ilz se \edtext{departirent}{\variant{partirent atw}} \edtext{tous}{\variant{$\phi$ atw brn}} d' Avignon et prindrent le chemin de \edtext{Nismes}{\variant{Nimes atw}} et de Montpellier\edtext{. Si chevauchierent}{\group{et chevaucerent brn; et chevaulcherent atw}} a petites journees et a grans \edtext{frais. Si}{\variant{fraiz et atw brn}} passerent Besiers et vindrent a \edtext{Carcassonne}{\variant{}}\edtext{. Et la}{\variant{ou atw; et brn}} \edtext{}{\variant{ilz atw}} trouverent \edtext{ilz}{\variant{$\phi$ atw brn}} messire Loÿs de Xancerre, mareschal de France,\edtext{\edtext{quy moult grandementles recueilly, ce fut raison.}{\variant{qui les recueillit grandement atw brn}}}{\explan{Enluminure ou miniature, signalée mais non consultable depuis Online Froissart}} \edtext{ \edtext{Et}{\subvariant{, et brn}} parla a \edtext{eulz}{\subvariant{eulx brn}} assez du conte de \edtext{Fois,}{\subvariant{Foix brn}} de son estat et de son affaire. Car il y avoit esté \edtext{de Carcassonne }{\subvariant{$\phi$ brn}} n'avoit point passé deux \edtext{mois}{\subvariant{moix brn}} }{\variant{$\phi$ atw}}.\edtext{\edtext{Ilz}{\group{Ilz atw; De Carcassonne, il brn}}}{\explan{Changement de page dans la leçon principale : passage au folio 371 v.}} se \edtext{departirent}{\group{partyrent brn; partirent atw}} \edtext{}{\variant{de la atw}} et vindrent \edtext{en la cité de}{\variant{a atw brn}} Thoulouse\edtext{,}{\variant{$\phi$ atw brn}} et la s'arresterent\edtext{. Et envoierent}{\variant{et envoyerent atw brn}} leurs \edtext{messages}{\variant{messagiers atw brn}} \edtext{}{\variant{devant atw}} \edtext{devers}{\variant{deverz brn}} le \edtext{ conte de Fois }{\variant{ conte de Foix atw brn}}\edtext{, quy}{\variant{qui brn;, qui atw}} se tenoit a Orthaiz en \edtext{Berne}{\variant{Bierne atw brn}}. Si \edtext{s'entamerent}{\variant{s'entammerent atw}} les traittiéz de ce mariage, mais ilz \edtext{}{\variant{y atw brn}} furent moult \edtext{loingtains}{\variant{longuement atw brn}}\edtext{, car}{\variant{. Car brn}} \edtext{de}{\variant{du atw brn}} commencement le conte de \edtext{Fois}{\variant{Foix atw brn}} fut \edtext{}{\variant{moult atw brn}} froit \edtext{et dur}{\variant{$\phi$ atw brn}} pour \edtext{ce}{\variant{tant brn}} que le duc de Lancastre\edtext{,}{\variant{$\phi$ brn}} \edtext{quy}{\variant{qui atw brn}} \edtext{pour ce temps se tenoit}{\variant{se tenoit pour le temps a atw brn}} a \edtext{Bourdeaulx}{\variant{Bourdeaulz atw}} ou a \edtext{ Liebourne, en}{\variant{Lierbonne, atw; Narbonne brn}} faisoit parler et \edtext{pryer}{\variant{prier atw brn}} pour son filz \edtext{messire}{\variant{$\phi$ atw brn}} Henry, conte \edtext{d'Erby.}{\variant{d'Erbi, atw}} \edtext{Si}{\variant{si atw; Sy brn}} fut \edtext{telle fois}{\variant{tele foiz brn}} \edtext{pour}{\variant{que brn}} le \edtext{loingtain}{\variant{longtain atw}} sejour \edtext{que l'}{\variant{qu' atw brn}} on veoit \edtext{advenir, que on disoit que le mariage pour lequel ces seigneurs s'arresterent en la bonne cité de}{\variant{a ces seigneurs faire qu'on disoit que le mariage pour lequel ilz sejournoient a atw brn}} Thoulouse\edtext{,}{\variant{$\phi$ atw brn}} ne se feroit point\edtext{, et}{\variant{. Et atw brn}} tout leur estat et \edtext{toutes}{\variant{$\phi$ atw brn}} les ordonnances, \edtext{responses}{\variant{responces atw brn}} et \edtext{traittiés}{\variant{traittiéz atw; traitiéz brn}} du conte de \edtext{Fois,}{\variant{Foix atw brn}} de jour en jour \edtext{et de septmaine en septmaine,}{\variant{, de sepmaine en sepmaine atw brn}} ilz \edtext{envoierent}{\variant{envoioient atw; envoyoient brn}} soingneusement devers le duc de Berry\edtext{,}{\variant{$\phi$ brn}} \edtext{quy}{\variant{qui atw brn}} se tenoit a la Nonnette en Auvergne. Et le duc\edtext{, quy}{\variant{qui atw brn}} n'avoit autre desir fors que les choses approchassent,\edtext{}{\variant{souvent atw}} \edtext{escripsoit}{\variant{$\phi$ atw; rescripvoit brn}} devers \edtext{eulz}{\variant{eulx brn}}\edtext{,}{\variant{$\phi$ atw brn}} et \edtext{les raffreschissoit souvent}{\variant{souvent les raffreschissoit de atw}} \edtext{nouveaulz}{\variant{nouveaulx atw brn}} \edtext{messages}{\variant{messaiges atw}}\edtext{. En}{\variant{en atw brn}} \edtext{eulz}{\variant{leur atw; eulx brn}} \edtext{signiffiant}{\variant{seigniffiant atw}} que \edtext{par nul moyen}{\variant{nullement atw brn}} \edtext{ilz}{\variant{$\phi$ atw}} ne \edtext{cessassent point}{\variant{cessasent atw}} que la \edtext{besoingne}{\variant{chose atw; besongne brn}} ne se \edtext{fesist.}{\variant{feïst. brn}} Le conte de \edtext{Foiz}{\variant{Foix atw brn}}, \edtext{quy}{\variant{qui atw brn}} estoit \edtext{sage}{\variant{saige atw}} \edtext{}{\variant{homme atw brn}} et \edtext{soubtil}{\variant{subtil brn}}\edtext{, et quy}{\variant{et qui atw brn}} veoit \edtext{l'ardent}{\variant{l'ardant atw brn}} desir du duc de Berry\edtext{, traittoit sagement}{\variant{traitoit saigement atw}} et bellement et si froidement mena \edtext{ses}{\variant{le atw; ces brn}} procés que par \edtext{l'accord}{\variant{accord atw brn}} de tous\edtext{, et}{\variant{et atw brn}} \edtext{encoires}{\variant{encores atw}} a \edtext{grans prieres}{\variant{grant priere atw brn}} il \edtext{ot}{\variant{eut atw; eust brn}} \edtext{trente mille}{\variant{XXXM atw brn}} frans pour les \edtext{dix}{\variant{$\phi$ atw brn}} ans \edtext{que il }{\variant{qu'il atw brn}} \edtext{avoit}{\variant{l'avoit atw}} \edtext{gardé}{\variant{gardee atw brn}} la damoiselle \edtext{}{\variant{, atw brn}} \edtext{nourrie et tenu son estat.}{\variant{nourie et tenue son estat atw}} \edtext{ Encoires se plus \edtext{il}{\subvariant{$\phi$ brn}} en eust demandé\edtext{,}{\subvariant{$\phi$ brn}} plus en eust eu et eut plus eul s'il eut demandé }{\variant{et eut plus eul s'il eut demandé atw}}\edtext{. Mais}{\variant{mais atw;, mais brn}} \edtext{moiennement il volt}{\variant{moyennement il en voult atw brn}} ouvrer \edtext{sur}{\variant{sus brn}} la conclusion de ceste matiere affin \edtext{que on luy}{\variant{qu'on lui atw brn}} en sceust gré\edtext{. Et}{\variant{et atw brn}} \edtext{aussi}{\variant{$\phi$ atw}} que le duc de \edtext{Berry}{\variant{Berri atw}} sentist que il \edtext{faisoit}{\variant{feist atw; feïst brn}} aucune chose pour \edtext{luy}{\variant{lui atw}}.
            \pend
            \endnumbering
            
            \pagebreak
            \tableofcontents
            \end{document}
        