 
            \documentclass[12pt, a4paper]{report}
            \usepackage[utf8x]{inputenc}
            \usepackage[T1]{fontenc}
            \usepackage{lmodern}
            \usepackage{graphicx}
            \usepackage[french]{babel}
            \usepackage{reledmac}
            \usepackage[switch, modulo]{lineno}
            
            % définition de commandes spécifiques pour l'apparat
            \newcommandx{\variant}[2][1,usedefault]{\Afootnote[#1]{#2}}
            \newcommandx{\explan}[2][1,usedefault]{\Bfootnote[#1]{#2}}
            
            \Xarrangement[A]{paragraph}
            \Xparafootsep{$\parallel$~}
            
            \begin{document}
            
            
            
            
            
            \beginnumbering
            \linenumbers
            \pstart
            
  
  
  
  
  
  
  
    
      
        Leduc de\edtext{Berry}{\variant{Berri  atw}}
        
        \edtext{,}{\variant{$\phi$ brn}}
        madame\edtext{Jehanne}{\variant{Jehenne  brn}}
          \edtext{d’Armeignach}{\variant{d’Ermignac  atw, d’Armignac  brn}}
        
        \edtext{sa premiere femme trespassee}{\variant{}}
        \edtext{,}{\variant{$\phi$ brn}}avoit grant\edtext{ymagination}{\variant{ymaginacion  atw  brn}}
        \edtext{. Et bien le}{\variant{et bien le  atw  brn}}
        \edtext{moustra}{\variant{monstra  atw  brn}}que secondement il\edtext{fust}{\variant{feust  atw, fut  brn}}
        \edtext{remarié}{\variant{marié  atw  brn}}. Car\edtext{tantost que}{\variant{}}il\edtext{pot}{\variant{peut  atw  brn}}veoir\edtext{comment il}{\variant{qu’il  atw  brn}}avoit\edtext{failly}{\variant{faily  atw, failli  brn}}
        \edtext{a la fille de Castille,}{\variant{$\phi$ atw  brn}}il ne fut oncques\edtext{aise}{\variant{avisé  atw}}ne\edtext{n’ot repos}{\variant{}}
        \edtext{. Mais}{\variant{mais  brn}}
        mist\edtext{en oeuvre clers}{\variant{}}et\edtext{messagiers}{\variant{messaigiers  atw  brn}}pour\edtext{envoier}{\variant{envoyer  atw  brn}}devers leconte\edtext{Gascoing}{\variant{Gascon  atw  brn}}de\edtext{Fois}{\variant{Foix  atw  brn}}
        
        \edtext{,}{\variant{$\phi$ brn}}
        \edtext{quy}{\variant{qui  atw  brn}}avoit en garde
          \edtext{la fille}{\variant{le filz  brn}}auconte Jehan de\edtext{Boulongne}{\variant{Bouloingne  atw}}
          
        et avoit\edtext{ja}{\variant{$\phi$ atw  brn}}eue plus deans. Et pour tant que leduc de\edtext{Berry}{\variant{Berri  atw}}
        ne pouoit\edtext{a ce mariage parvenir}{\variant{parvenir a ce mariage  atw}}fors par\edtext{les dangiers}{\variant{le dangier  atw  brn}}duconte de\edtext{Fois}{\variant{Foix  atw  brn}}
        . Car au fort le ditcontene pour perene pour mere
        \edtext{ne}{\variant{$\phi$ brn}}pour papene pour amy que ladamoiselleeust\edtext{,}{\variant{$\phi$ brn}}il n’en\edtext{euist}{\variant{eust  atw  brn}}riens faitvenu a\edtext{sa}{\variant{$\phi$ atw  brn}}plaisance. Il en parla auroy de France, son nepveu, et au duc de\edtext{Bourgoingne}{\variant{Bourgongne  brn}}
        , son frere\edtext{. Et leur}{\variant{, et leur  atw  brn}}
        \edtext{requist moult}{\variant{pria  atw  brn}}affectueusement\edtext{que ilz}{\variant{qu’ilz  atw  brn}}s’en voulsissent chargier\edtext{, avecques luy et embesoingnier}{\variant{}}. Leroy de Franceen\edtext{ot}{\variant{eut  atw  brn}}bon\edtext{rys,}{\variant{ris  atw  brn}}pour tant que leduc de Berryestoit ja tout\edtext{anchien,}{\variant{ancien  atw  brn}}et\edtext{luy}{\variant{lui  atw  brn}}
        \edtext{dist}{\variant{dit  atw}}:\edtext{"Beaulx oncles}{\variant{"Bel oncle  atw  brn}}, que feréz vous
        n'a que\edtext{douze}{\variant{XII  atw  brn}}ans et vous en avéz\edtext{soixante}{\variant{LX  atw  brn}}
        \edtext{.}{\variant{$\phi$ atw}}Par ma foy,\edtext{c’est}{\variant{il me semble  atw}}grant\edtext{folie pour}{\variant{follie de  atw}}vous de penser a telle\edtext{besoigne}{\variant{chose  atw  brn}}.\edtext{Faittes}{\variant{Faictes  brn}}
        \edtext{ent}{\variant{en  atw}}parler pourJehan
        beau cousin vostre filz,\edtext{quy}{\variant{qui  atw  brn}}est\edtext{jenne}{\variant{jeune  atw}}et a venir\edtext{. La}{\variant{, la  atw}}chose est\edtext{trop}{\variant{$\phi$ atw  brn}}mieulx pareille a\edtext{\edtext{luy}{\variant{lui  atw  brn}}}{\explan{Enluminure ou miniature, signalée mais non consultable depuis Online Froissart}}que a vous." "Monseigneur,"\edtext{respondi}{\variant{respondit  atw  brn}}leduc de\edtext{Berry}{\variant{Berri  atw}}
        , "on en a parlé\edtext{,}{\variant{$\phi$ atw  brn}}mais leconte de\edtext{Foiz}{\variant{Foix  atw  brn}}
        n’y\edtext{a voulu}{\variant{}}entendre pour la cause de ce quemon filzdescend de ceulx\edtext{d’Armeignach. Et}{\variant{d’Armignac et  atw, de Armignac  brn}}ilz sont en guerre et en hayne et ont esté\edtext{long temps}{\variant{longtemps  brn}}ensemble. Se lafilleest\edtext{jenne}{\variant{jeune  atw}}, je\edtext{la espargneray}{\variant{}}
        \edtext{trois}{\variant{troiz  brn}}ou quatre ans\edtext{, voire si longuement}{\variant{tant  atw  brn}}qu’elle sera femme\edtext{parfaitte et}{\variant{parfaicte  brn}}
        ." Voire,"leroy,
        \edtext{mais}{\variant{maiz  atw}}elle ne vous\edtext{espargnera}{\variant{espareignera  atw}}pas." Et puis dist leroytout en\edtext{ryant}{\variant{riant  atw  brn}}:,\edtext{puisque}{\variant{puis que  atw  brn}}nous veons que\edtext{si tres}{\variant{tant  atw, si  brn}}grande affection vous y avéznous y\edtext{entendrons}{\variant{entenderons  brn}}
        \edtext{moult}{\variant{$\phi$ atw  brn}}voulentiers."ne demoura\edtext{gaires}{\variant{guerres  atw}}de temps que leroyordonna lesire de la Riviere, messire\edtext{Burel}{\variant{Brueil  atw  brn}}
        , son souverain\edtext{chevallier}{\variant{$\phi$ atw  brn}}
        \edtext{et}{\variant{$\phi$ brn}}maistre d’ostel et\edtext{chambellan}{\variant{chambrelan  atw, chambrelain  brn}}
        \edtext{,}{\variant{$\phi$ atw}}pour\edtext{aler}{\variant{aller  atw}}en ce voyage\edtext{. Et avecques}{\variant{et avecques  atw  brn}}
        \edtext{luy}{\variant{lui  atw  brn}}leconte\edtext{d’Assy}{\variant{d’Acy  atw  brn}}
        . Et leduc de\edtext{Bourgoingne}{\variant{Bourgongne  brn}}
        y ordonna pour\edtext{luy}{\variant{lui  atw  brn}}et en son noml’evesque\edtext{d’Authun}{\variant{d’Autun  brn}}
        et>messire\edtext{Guillemme}{\variant{Guillaume  atw, Guillame  brn}}de la Trimoulle
        \edtext{et}{\variant{. Et  atw  brn}}le>duc de Berryy ordonna\edtext{et}{\variant{$\phi$ atw}}en\edtext{prya}{\variant{pria  atw  brn}}le conteJehan de Xancerre
        \edtext{ung}{\variant{, un  brn}}grant\edtext{, sage}{\variant{saige  atw  brn}}et vaillant\edtext{chevallier}{\variant{chevalier  atw}}.Ces\edtext{cinq}{\variant{cincq  atw}}seigneurs pour venir\edtext{devers}{\variant{vers  atw  brn}}leconte de\edtext{Fois}{\variant{Foix  atw  brn}}
        etrequerir\edtext{en}{\variant{ou  atw  brn}}nom\edtext{de mariage}{\variant{}}pour leduc de Berryceste\edtext{jenne}{\variant{jeune  atw}}
        dame
        
        \edtext{,}{\variant{$\phi$ brn}}se partirent de leurs lieux et se devoient trouver enAvignon
        
        \edtext{. Et environ la}{\variant{, et environ la  atw  brn}}
        \edtext{Chandelleur}{\variant{Chandeller  atw, Chandeleur  brn}}ilz se\edtext{departirent}{\variant{partirent  atw}}
        \edtext{tous}{\variant{$\phi$ atw  brn}}d’Avignonet prindrent le chemin de
          \edtext{Nismes}{\variant{Nimes  atw}}
        et deMontpellier
        \edtext{. Si chevauchierent}{\variant{}}a petites journees et a grans\edtext{frais. Si}{\variant{fraiz et  atw  brn}}passerentBesierset vindrent a
          \edtext{Carcassonne}{\variant{}}
        
        \edtext{. Et la}{\variant{ou  atw, et  brn}}
        trouverent\edtext{ilz}{\variant{$\phi$ atw  brn}}messireLoÿs de Xancerre, mareschal de France,\edtext{\edtext{quy moult grandementles recueilly, ce fut raison.}{\variant{qui les recueillit grandement  atw  brn}}}{\explan{Enluminure ou miniature, signalée mais non consultable depuis Online Froissart}}
        .se\edtext{departirent}{\variant{}}
        et vindrent\edtext{en la cité de}{\variant{a  atw  brn}}
        Thoulouse
        \edtext{,}{\variant{$\phi$ atw  brn}}et la s’arresterent\edtext{. Et envoierent}{\variant{et envoyerent  atw  brn}}leurs\edtext{messages}{\variant{messagiers  atw  brn}}
        
        \edtext{devers}{\variant{deverz  brn}}le
        \edtext{,}{\variant{$\phi$ brn}}
        \edtext{quy}{\variant{qui  atw  brn}}se tenoit aOrthaiz en\edtext{Berne}{\variant{Bierne  atw  brn}}
        . Si\edtext{s’entamerent}{\variant{s’entammerent  atw}}les traittiéz de ce mariage, mais ilzfurent moult\edtext{loingtains}{\variant{longuement  atw  brn}}
        \edtext{, car}{\variant{. Car  brn}}
        \edtext{de}{\variant{du  atw  brn}}commencement leconte de\edtext{Fois}{\variant{Foix  atw  brn}}
        futfroit\edtext{et dur}{\variant{$\phi$ atw  brn}}pour\edtext{ce}{\variant{tant  brn}}que leduc de Lancastre
        \edtext{,}{\variant{$\phi$ brn}}
        \edtext{quy}{\variant{qui  atw  brn}}
        \edtext{pour ce temps se tenoit}{\variant{se tenoit pour le temps a  atw  brn}}a
          \edtext{Bourdeaulx}{\variant{Bourdeaulz  atw}}
        ou afaisoit parler et\edtext{pryer}{\variant{prier  atw  brn}}pour son filz\edtext{messire}{\variant{$\phi$ atw  brn}}
        Henry, conte\edtext{d’Erby.}{\variant{d’Erbi,  atw}}
        
        \edtext{Si}{\variant{si  atw, Sy  brn}}fut\edtext{telle fois}{\variant{tele foiz  brn}}
        \edtext{pour}{\variant{que  brn}}le\edtext{loingtain}{\variant{longtain  atw}}sejour\edtext{que l'}{\variant{qu'  atw  brn}}on veoit\edtext{advenir, que on disoit que le mariage pour lequel ces seigneurs s’arresterent en la bonne cité de}{\variant{a ces seigneurs faire qu’on disoit que le mariage pour lequel ilz sejournoient a  atw  brn}}
        Thoulouse
        \edtext{,}{\variant{$\phi$ atw  brn}}ne se feroit point\edtext{, et}{\variant{. Et  atw  brn}}tout leur estat et\edtext{toutes}{\variant{$\phi$ atw  brn}}les ordonnances,\edtext{responses}{\variant{responces  atw  brn}}et\edtext{traittiés}{\variant{traittiéz  atw, traitiéz  brn}}duconte de\edtext{Fois,}{\variant{Foix  atw  brn}}
        de jour en jour\edtext{et de septmaine en septmaine,}{\variant{, de sepmaine en sepmaine  atw  brn}}ilz\edtext{envoierent}{\variant{envoioient  atw, envoyoient  brn}}soingneusement devers leduc de Berry
        \edtext{,}{\variant{$\phi$ brn}}
        \edtext{quy}{\variant{qui  atw  brn}}se tenoit ala Nonnette en Auvergne. Et leduc
        \edtext{, quy}{\variant{qui  atw  brn}}n’avoit autre desir fors que les choses approchassent,
        \edtext{escripsoit}{\variant{$\phi$ atw, rescripvoit  brn}}devers\edtext{eulz}{\variant{eulx  brn}}
        \edtext{,}{\variant{$\phi$ atw  brn}}et\edtext{les raffreschissoit souvent}{\variant{souvent les raffreschissoit de  atw}}
        \edtext{nouveaulz}{\variant{nouveaulx  atw  brn}}
        \edtext{messages}{\variant{messaiges  atw}}
        \edtext{. En}{\variant{en  atw  brn}}
        
        \edtext{signiffiant}{\variant{seigniffiant  atw}}que\edtext{par nul moyen}{\variant{nullement  atw  brn}}
        \edtext{ilz}{\variant{$\phi$ atw}}ne\edtext{cessassent point}{\variant{cessasent  atw}}que lane se\edtext{fesist.}{\variant{feïst.  brn}}Leconte de\edtext{Foiz}{\variant{Foix  atw  brn}}
        ,\edtext{quy}{\variant{qui  atw  brn}}estoit\edtext{sage}{\variant{saige  atw}}
        et\edtext{soubtil}{\variant{subtil  brn}}
        \edtext{, et quy}{\variant{et qui  atw  brn}}veoit\edtext{l’ardent}{\variant{l’ardant  atw  brn}}desir duduc de Berry
        \edtext{, traittoit sagement}{\variant{traitoit saigement  atw}}et bellement et si froidement menaprocés que par\edtext{l’accord}{\variant{accord  atw  brn}}de tous\edtext{, et}{\variant{et  atw  brn}}
        \edtext{encoires}{\variant{encores  atw}}a\edtext{grans prieres}{\variant{grant priere  atw  brn}}
        il
        \edtext{ot}{\variant{eut  atw, eust  brn}}
        \edtext{trente mille}{\variant{XXXM  atw  brn}}frans pour les\edtext{dix}{\variant{$\phi$ atw  brn}}ans
        \edtext{avoit}{\variant{l’avoit  atw}}
        \edtext{gardé}{\variant{gardee  atw  brn}}ladamoiselle
        
        \edtext{nourrie et tenu son estat.}{\variant{nourie et tenue son estat  atw}}
        
        \edtext{. Mais}{\variant{mais  atw, , mais  brn}}
        \edtext{moiennement il volt}{\variant{moyennement il en voult  atw  brn}}ouvrer\edtext{sur}{\variant{sus  brn}}la conclusion de ceste matiere affin\edtext{que on luy}{\variant{qu’on lui  atw  brn}}en sceust gré\edtext{. Et}{\variant{et  atw  brn}}
        \edtext{aussi}{\variant{$\phi$ atw}}que leduc de\edtext{Berry}{\variant{Berri  atw}}
        sentist que il\edtext{faisoit}{\variant{feist  atw, feïst  brn}}aucune chose pour
          \edtext{luy}{\variant{lui  atw}}
        .
    
  

            \pend
            \endnumbering
            \end{document}
        